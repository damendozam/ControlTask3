\documentclass[12pt]{article}
%%%%%%%%%%%%%%%%%%%%%%%%%%%%%%%%%%%%%%%%%%%%%%%%%%%%%%%%%%%%%%%%%%%%%%%%%%%%%%%%%%%%%%%%%%%%%%%%%%%%%%%%%%%%%%%%%%%%%%%%%%%%
\usepackage{graphicx}
\usepackage{lmodern}
\usepackage{amsmath}
\usepackage{amsfonts}
\usepackage[spanish]{babel}
\usepackage[latin1]{inputenc}
\usepackage{hyperref}
\usepackage{fancyhdr}
\usepackage{pst-all}
%\usepackage{pstcol}
\setlength{\textwidth}{16.cm}
\setlength{\oddsidemargin}{0cm}
\setlength{\evensidemargin}{0cm}
\setlength{\textheight}{22cm}
\setlength{\headheight}{13.2pt}
\setlength{\headsep}{10mm}
\setlength{\marginparwidth}{30mm}
\setlength{\topmargin}{-.5cm}
\newcommand{\Cen}{\mathbb{C}^{n}}
\newcommand{\Com}{\mathbb{C}}
\newcommand{\Rn}{\mathbb{R}^{n}}
\newcommand{\erre}{\mathbb{R}}
\newcommand{\grados}{^{\circ}}
\newcommand{\dB}{\text{ dB}}
\newcommand{\radseg}{\text{ rad/seg}}
\newcommand{\om}{\omega}
\newcommand{\RHi}{RH_{\infty}}
\newcommand{\BM}{\begin{bmatrix}}
\newcommand{\EM}{\end{bmatrix}}
\DeclareMathOperator{\rank}{rank}
\DeclareMathOperator{\Imag}{Im}
\title{ControlRobusto}
\author{Hernando D�az Morales, Ph. D.}
\date{}
\newcounter{TareaNo}
\setcounter{TareaNo}{3}
%
% Comandos para generar el logo de la Universidad Nacional
% Definimos el color del logo
\newgray{ungris}{0.6}
%
% Dimension del logo 
\newdimen\UNlong
%
% Genera la U
\newcommand{\UNU}{%
 	\pscustom{
 		\psline[linearc=0.45]{C-C}(0,1)(0,0)(1.102,0)
		\psline(1.102,1)(0.92,1)(0.92,0.487)
		\psline[linearc=0.35](0.92,0.152)(0.553,0.152)(0.153,0.152)(0.153,0.487)
		\psline(0.153,1)(0,1)
	\fill[linecolor=ungris,fillstyle=solid,fillcolor=ungris]}}
%
% Genera la UN
\newcommand{\UNlogo}[1]{
	\UNlong=#1
	\psset{unit=1\UNlong,linewidth=0.1pt,linecolor=ungris}
  \begin{pspicture}(-0.1,-0.1)(2.386,1.1)
 		\rput(0,0){\UNU}
		\rput{180}(2.286,1){\UNU}
	\end{pspicture}
}
% fin definicion de UNlogo
%
%% Definicion del formato de los encabezados de p�gina
\pagestyle{fancy}
\newsavebox{\UNl}
\savebox{\UNl}{\UNlogo{3mm}}
\lhead{\usebox{\UNl}\scshape\scriptsize Control Robusto --- 2021}
\chead{}
\rhead{\scshape\scriptsize Tarea \theTareaNo}
\lfoot{}
\cfoot{\sffamily\small\thepage}
\rfoot{}
% Para eliminar la linea del encabezado, quitar el % de la siguiente linea
\renewcommand{\headrulewidth}{0.25pt}
%%
\begin{document}
\setcounter{TareaNo}{3}
\begin{center}
\textbf{Control Robusto --- Tarea \theTareaNo}\\
Para el 27 de mayo, 2021 (tentativamente)
\end{center}

\begin{enumerate}
\item \label{prob:matr} Calcule la norma 2 y la norma $\infty$\ de las siguientes matrices de transferencia:
\begin{enumerate}
	\item $G_1(s)=\left[\begin{array}{cc}
		\frac{1}{s+1}&\frac{s+3}{\left(s+1\right)\left(s-2\right)}\\
		\frac{10}{s-2}&\frac{5}{\left(s+1\right)}\end{array}\right]$
	\item $G_2(s)=\left[\begin{tabular}{cc|c}1&0&1\\2&3&1\\\hline1&2&0\end{tabular}
\right]$
	\item $G_3(s)=\left[\begin{tabular}{cc|c}-1&-2&1\\1&0&0\\\hline2&3&0\end{tabular}
\right]$
	\item $G_4(s)=\left[\begin{tabular}{ccc|cc}-1&-2&-3&1&2\\
	1&0&0&0&1\\0&1&0&2&0\\\hline1&0&0&1&0\\0&1&1&0&2\end{tabular}\right]$
\end{enumerate}

\item Sean $Q$ y $R$ matrices complejas $n\times n$ y suponga que $Q$ es no singular. Demuestre que, si, 
$ \overline{\sigma}(R) < \underline{\sigma}(Q)$, entonces $Q + R$ es no singular.

\textbf{Sug.:} Use las propiedades de los valores singulares.

\item Suponga que $\Delta$ satisface $\overline{\sigma}\left(\Delta\right)=\left\|\Delta\right\|<1$. Demuestre que 
		\begin{enumerate}
			\item $I-\Delta$ es nosingular
			\item La serie $\sum_{k=0}^{\infty}\Delta^k$ converge.
			\item $\sum_{k=0}^{\infty}\Delta^k=\left(I-\Delta\right)^{-1}$
		\end{enumerate}
\item  En el sistema de la figura siguiente, suponga que $H$ y $\Delta $
ambos per\-te\-ne\-cen a $\RHi$.

\begin{center}
\vspace{0.6cm}
\psset{xunit=0.3cm}
	\psset{yunit=0.25cm}
	\psset{runit=0.25cm}
	\psset{linewidth=1pt}
	\begin{pspicture}(0,0)(22,10)
	\psframe(8,1)(12,4)
	\psframe(8,7)(12,10)
	\pscircle(3.5,2.5){1}
	\pscircle(16.5,8.5){1}
	\psline{->}(0.5,2.5)(2.5,2.5)
	\psline{->}(4.5,2.5)(8,2.5)
	\psline{->}(12,2.5)(16.5,2.5)(16.5,7.5)
	\psline{->}(20.5,8.5)(17.5,8.5)
	\psline{->}(15.5,8.5)(12,8.5)
	\psline{->}(8,8.5)(3.5,8.5)(3.5,3.5)
	\uput[180](0.5,2.5){$d_2$}
	\uput[0](20.5,8.5){$d_1$}
	\uput[90](14,8.5){$v_1$}
	\uput[90](6,8.5){$v_2$}
	\rput(10,8.5){$\Delta$}
	\rput(10,2.5){$H$}
	\end{pspicture}		
\end{center}


\begin{enumerate}
\item  Pruebe que el sistema es estable E/S (es decir, la funci\'{o}n de
transferencia de $\begin{bmatrix}
d_1\\d_2\end{bmatrix} \mapsto \begin{bmatrix}
v_1\\v_2\end{bmatrix} $ es $\RHi$) si y 
s�lo si $%
I-\Delta H$ es invertible \ en $\RHi$. \textbf{Sug.:} Use el teorema de estabilidad.

\item  Suponga que $\left\| \Delta \right\| _{\infty }\left\| H\right\|
_{\infty }<1$. Demuestre que, entonces, el sistema es estable.
\end{enumerate}

\item  Sea $G(s)=\dfrac{s-2}{\left( s+3\right) \left( s-1\right) }$, Halle
una factorizaci\'{o}n coprima $G=nm^{-1}$ y $x,y\in \RHi$,
tales que $xn+ym=1$. 

Sean 
\[
	N(s)=\dfrac{\left(s-1\right)\left(s+\alpha\right)}{\left(s+2\right)\left(s+3\right)\left(s+\beta\right)};
	\qquad
	M(s)=\dfrac{\left(s-3\right)\left(s+\alpha\right)}{\left(s+3\right)\left(s+\beta\right)}
\]
Pruebe que $(N,M)$ tambi�n es una factorizaci�n coprima para cualquier $\alpha>0$ y $\beta>0$.
\item  Sea  $P(s)=\dfrac{1}{s-1}$. Encuentre el conjunto de todos los controladores que estabilizan a esta planta. A continuaci�n, verifique que  $K=-4$ es un controlador estabilizador y encuentre el factor $Q$ que da este controlador, a partir de su controlador original.
\item Encuentre todos los controladores que estabilizan la planta
\[
G(s)=\dfrac{s-1}{(s-2)(s^2 +1)}
\]
\item Un sistema MIMO tiene la matriz de transferencia 
\[
G(s)=\begin{bmatrix}\dfrac{-5}{25s+1}&\dfrac{s^2-0.005s-0.005}{s(s+1)}\\[4mm]
\dfrac{1}{25s+1}&\dfrac{-0.0023}{s} \end{bmatrix}
\]
\begin{enumerate}
\item Encuentre una factorizaci�n doblemente coprima en $\RHi$ para esta planta.
\item Halle un controlador que estabilice a la planta.
\end{enumerate}
\item Un modelo para un reactor ``batch'' inestable, descrito por Rosenbrock puede ser linealizado en la forma:
\[
	\dot{x}=\begin{bmatrix}
	1.38& -0.2077 &  6.715 &  -5.676\\
	-0.5814 & -4.29 & 0 & 0.675\\
	1.067 & 4.273 & 1.343 & -2.104\\
	0.048 & 4.273 & 1.343 & -2.104
	\end{bmatrix}\, x +
	\begin{bmatrix}
	0 & 0\\
	5.679 & 0\\
	1.136 & -3.146\\
	1.136 & 0
	\end{bmatrix}\, u
\]
\[
	y=\begin{bmatrix} 1& 0& 1& -1\\
	0 & 1 & 0& 0
	\end{bmatrix}\, x
\]
\begin{enumerate}
\item Encuentre factorizaciones doblemente coprima en $\RHi$ para este sistema
\item Encuentre un controlador que estabilice a este sistema, basado en la factorizaci�n hallada. Muestre que efectivamente estabiliza al sistema.
\item Muestre que el controlador proporcional integral siguiente estabiliza al sistema tambi�n.
\[
	K(s) = -\begin{bmatrix} 0&2\\-5&0\end{bmatrix}-\dfrac{1}{s}\begin{bmatrix} 0&2\\-8&0\end{bmatrix}
\]
\item Muestre que el controlador anterior se puede escribir como una parametrizaci�n del controlador hallado arriba. 
\end{enumerate}

\end{enumerate}
\begin{thebibliography}{9}
\bibitem{Green}M. Green. D.J.N. Limebeer.  \textit{Linear Robust Control.} Prentice Hall, 1995. 1995.
\end{thebibliography}

\end{document}

